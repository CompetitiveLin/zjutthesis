% !Mode:: "TeX:UTF-8"
%% 英文摘要
\begin{abstracte}
	
	Rubik’s cube is a rubik’s cube with an octahedral structure. With the continuous deepening of research on the Rubik’s cube and the increasing evolution of the rubik’s
	Cube reduction technology, the Rubik’s cube has become an easy-to-use, low-threshold puzzle toy. But at the same time, considering the human brain in memory cube color and
	solving the existence of certain limitations, the use of fingers to rotate magic compared to the motor driven magic rotation in rotation speed there are fundamental differences.
	
	In view of this problem, this paper proposes a set of software and hardware co-design scheme to achieve high efficiency and fast restoration of the three-order Rubik’s cube
	restoration system. The specific research contents are as follows:
	
	1. Designed and implemented a six-axis architecture for Rubik’s Cube restoration. Traditional two-axis and four-axis rubik’s Cube restoration systems often have disadvantages such as redundancy of restoration steps. The architecture proposed in this paper adopts the design of increasing the number of rotation axes to improve the efficiency of rubik’s cube restoration. On this basis, this paper independently designed the PCB which is connected with the lower machine and the stepper motor, so as to carry the lower machine and transmit the motor rotation signal.
	
	2. The rubik’s cube rotation of parallel restoration algorithm is proposed, and the more mature - Kociemba rubik’s cube recovery algorithm is presented on the basis of the rubik’s cube recovery sequence of steps in each other the rotation of the parallel operation, with the existing serial principle compared to rotate in order to complete the recovery method, the method of this paper can effectively improve the speed of the rubik’s cube recovery.
	
	3. Designed and realized the visual interface of the system. It provides functions such as correcting the color of rubik’s cube and verifying the feasibility of restoration. It can effectively ensure the feasibility of the system by manually correcting the accidental deviation similar to color recognition error.
	
	The experimental results show that the system can restore a rubik’s cube in an arbitrarily disturbed state in 2 seconds on average, which exceeds most existing rubik’s cube restoration systems.
	
	\keywordse{rubik’s 3x3 cube, rubik’s cube restoration, software and hardware co-design,	parallel optimization}
\end{abstracte}
