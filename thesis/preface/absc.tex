% !Mode:: "TeX:UTF-8"
%% 中文摘要
\begin{abstractc}
	魔方是一种具有八面体结构的魔术方块。人们对魔方研究的不断深入,伴随着魔方还原技术的日益进化,魔方已经成为一种易上手、门槛低的益智类玩具。但同时又考虑到人脑在记忆魔方块颜色并求解时存在一定局限性,使用手指转动魔方面相比于电机带动魔方面转动在旋转速度上存在根本性的差异等问题。针对这一问题,本论文提出一套通过软硬件协同设计的方案,实现还原效率既高,还原速度又快的三阶魔方还原系统。具体研究内容如下:
	
	1. 设计并实现了面向魔方复原的六轴架构。传统二轴和四轴的魔方复原系统存在步骤冗余等缺点,本论文提出的架构采用增加旋转轴数的设计,提升魔方复原效率。在此基础上,本论文自主设计了与下位机、步进电机相连接的 PCB 电路板,从而能够搭载下位机并传递电机旋转信号。
	
	2. 提出了并行复原的魔方旋转算法,在较为成熟的魔方复原算法 —— Kociemba 算法的基础上,将魔方复原步骤序列中互不干扰的旋转操作并行化,与现有采用串行原则依次进行旋转完成复原的方法相比,本论文的方法能有效提升魔方复原的速度。
	
	3. 设计并实现了系统可视化界面。提供矫正魔方块颜色、验证还原可行性等功能,针对系统发生类似于颜色识别错误的偶然性偏差时能通过人工手动修正的方式有效保证系统可行性。
	
	实验结果表明,该系统能平均在 2 秒内还原一个任意打乱状态下的魔方,超越大多数现有的魔方复原系统。
	
\keywordsc{三阶魔方,魔方还原,软硬件协同设计,并行优化}
\end{abstractc}
