% !Mode:: "TeX:UTF-8"

\chapter{数学公式的输入方法}
\section{本科生毕业设计论文的公式规范}

论文中的公式应另起行,原则上应居中书写,与周围文字留有足够的空间区分开。
若公式前有文字(如“解”、“假定”等),文字空两格写,公式仍居中写。公式末不加标点。

公式应标注序号,并将序号置于括号内。 公式序号按章编排,如第~1~章第一个公式序号为“(1-1)”。公式的序号右端对齐。

公式较长时最好在等号“=”处转行,如难实现,则可在~$+$、$-$、$\times$、$\div$~运算符号处转行,转行时运算符号仅书写于转行式前,不重复书写。

文中引用公式时,一般用“见式~(1-1)”或“由公式~(1-1)”。

公式中用斜线表示“除”的关系时应采用括号,以免含糊不清,如~$a/(b\cos x)$。通常“乘”的关系在前,如~$a\cos x/b$而不写成~$(a/b)\cos x$。

不能用文字形式表示等式,如:$\textnormal{刚度}=\frac{{\textnormal{受力}}}{{\textnormal{受力方向的位移}}}$。

对于数学公式的输入方法,网络上有一个比较全面权威的文档\textbf{~\href{http://tug.ctan.org/cgi-bin/ctanPackageInformation.py?id=voss-mathmode}{Math mode}}~请大家事先大概浏览一下。下面将对学位论文中主要用到的数学公式排版形式进行阐述。

\section{生成~\XeLaTeX~数学公式的两种方法}
对于先前没有接触过~\XeLaTeX~的人来说,编写~\XeLaTeX~数学公式是一件很繁琐的事,尤其是对复杂的数学公式来说,更可以说是一件难以完成的任务。
实际上,生成~\XeLaTeX~数学公式有两种较为简便的方法,一种是基于~MathType~数学公式编辑器的方法,另一种是基于~MATLAB~商业数学软件的方法,
下面将分别对这两种数学公式的生成方法作一下简单介绍。

\subsection{基于~MathType~软件的数学公式生成方法}
MathType~是一款功能强大的数学公式编辑器软件,能够用来在文本环境中插入~Windows OLE~图形格式的复杂数学公式,所以应用比较普遍。但此软件只有~30~天的试用期,之后若再继续使用则需要付费购买才行。网络上有很多破解版的~MathType~软件可供下载免费使用,
笔者推荐下载安装版本号在~6.5~之上的中文破解版。

在安装好~MathType~之后,若在输入窗口中编写数学公式,复制到剪贴板上的仍然是图形格式的对象。
若希望得到可插入到~\XeLaTeX~编辑器中的文本格式对象,则需要对~MathType~软件做一下简单的设置:在~MathType~最上排的按钮中依次选择“参数选项
$\to$转换”,在弹出的对话窗中选中“转换到其它语言(文字):”,在转换下拉框中选择“Tex~--~--~LaTeX 2.09 and later”,并将对话框最下方的两个复选框全部勾掉,点击确定,这样,再从输入窗口中复制出来的对象就是文本格式的了,就可以直接将其粘贴到~\XeLaTeX~
编辑器中了。按照这种方法生成的数学公式两端分别有标记\verb|\[|和标记\verb|\]|,在这两个标记之间才是真正的数学公式代码。

若希望从~MathType~输入窗口中复制出来的对象为图形格式,则只需再选中“公示对象(Windows OLE~图形)”即可。

\subsection{基于~MATLAB~软件的数学公式生成方法}

MATLAB~是矩阵实验室(Matrix Laboratory)的简称,是美国~MathWorks~公司出品的商业数学软件。它是当今科研领域最常用的应用软件之一,
具有强大的矩阵计算、符号运算和数据可视化功能,是一种简单易用、可扩展的系统开发环境和平台。

MATLAB~中提供了一个~latex~函数,它可将符号表达式转化为~\XeLaTeX~数学公式的形式。其语法形式为~latex(s),其中,~s~为符号表达式,
之后再将~latex~函数的运算结果直接粘贴到~\XeLaTeX~编辑器中。从~\XeLaTeX~数学公式中可以发现,其中可能包含如下符号组合:

\begin{verbatim*}
\qquad=两个空铅(quad)宽度
\quad=一个空铅宽度
\;=5/18空铅宽度
\:=4/18空铅宽度
\,=3/18空铅宽度
\!=-3/18空铅宽度
\ =一个空格
\end{verbatim*}

所以最好将上述符号组合从数学公式中删除,从而使数学公式显得匀称美观。

对于~Word~等软件的使用者来说,在我们通过~MATLAB~运算得到符号表达式形式的运算结果时,在~Word~中插入运算结果需要借助于~MathType~软件,
通过在~MathType~中输入和~MATLAB~运算结果相对应的数学表达形式,之后再将~MathType~数学表达式转换为图形格式粘贴到~Word~中。实际上,
也可以将~MATLAB~中采用~latex~函数运行的结果直接粘贴到~MathType~中,再继续上述步骤,这样可以大大节省输入公式所需要的时间。
此方法在~MathType~6.5c~上验证通过,若您粘入到~MathType~中的仍然为从~MATLAB~中导入的代码,请您更新~MathType~软件。

\section{数学字体}
在数学模式下,常用的数学字体命令有如下几种:

\begin{verbatim}
\mathnormal或无命令 用数学字体打印文本;
\mathit             用斜体(\itshape)打印文本;
\mathbf             用粗体(\bfseries)打印文本;
\mathrm             用罗马体(\rmfamily)打印文本;
\mathsf             用无衬线字体(\sffamily)打印文本;
\mathtt             用打印机字体(\ttfamily)打印文本;
\mathcal            用书写体打印文本;
\end{verbatim}

在学位论文撰写中,只需要用到上面提到的~\verb|\mathit|、\verb|\mathbf|~和~\verb|\mathrm|~命令。若要得到~Times New Roman~的数学字体,则需要调用~txfonts~宏包(此宏包实际上采用的是~Nimbus Roman No9 L~字体,
它是开源系统中使用的免费字体,其字符字体与~Times New Roman~字体几乎完全相同);若要得到粗体数学字体,则需要调用~bm~宏包。表~\ref{tab:fonts}~中分别列出了得到阿拉伯数字、拉丁字母和希腊字母
各种数学字体的命令。

\begin{table}[htbp]
\caption{常用数学字体命令一览}\label{tab:fonts}
\vspace{0.5em}\centering\wuhao
\begin{tabular}{llll}
\toprule
 & 阿拉伯数字\&大写希腊字母 & 大小写拉丁字母 & 小写希腊字母  \\
\midrule
斜体 & \verb|\mathit{}| & \verb|无命令| & \verb|无命令|\\
粗斜体 & \verb|\bm{\mathit{}}| & \verb|\bm{}| & \verb|\bm{}|\\
直立体 & \verb|无命令| & \verb|\mathrm{}| & \verb|字母后加up|\\
粗体 & \verb|\mathbf{}或\bm{}| & \verb|\mathbf{}| & \verb|\bm{字母后加up}|\\
\bottomrule
\end{tabular}
\vspace{\baselineskip}
\end{table}

\noindent 下面列出了一些应采用直立数学字体的数学常数和数学符号。

\section{行内公式}
出现在正文一行之内的公式称为行内公式,例如~$f(x)=\int_{a}^{b}\frac{\sin{x}}{x}dx$。对于非矩阵和非多行形式的行内公式,一般不会使得行距发生变化,而~Word~等软件却会根据行内公式的竖直距离而自动调节行距,如图~\ref{fig:hangju}~所示。

\begin{figure}[htbp]
\centering
\subfigure[由~\XeLaTeX~系统生成的行内公式]{\label{fig:subfig:latex}
                \fbox{\includegraphics[width=0.55\textwidth]{latex}}}
\subfigure[由~Word软件生成的~.doc~格式行内公式]{\label{fig:subfig:word}
                \fbox{\includegraphics[width=0.55\textwidth]{word}}}
\subfigure[由~Word软件生成的~.pdf~格式行内公式]{\label{fig:subfig:pdf}
                \fbox{\includegraphics[width=0.55\textwidth]{pdf}}}

\caption{由~\XeLaTeX~和~Word~生成的~3~种行内公式屏显效果}\label{fig:hangju}
\vspace{-1em}
\end{figure}

这三幅图分别为~\XeLaTeX~和~Word~生成的行内公式屏显效果,从图中可看出,在~\XeLaTeX~文本含有公式的行内,在正文与公式之间对接工整,行距不变;而在~Word~文本含有公式的行内,在正文与公式之间对接不齐,行距变大。因此从这一点来说,
\XeLaTeX~系统在数学公式的排版上具有很大优势。

\XeLaTeX~提供的行内公式最简单、最有效的方法是采用~\TeX~本来的标记———开始和结束标记都写作~\$,例如本段开始的例子可由下面的输入得到。
\verb|$f(x)=\int_{a}^{b}\frac{\sin{x}}{x}\mathrm{d}x$|

\section{行间公式}
位于两行之间的公式称为行间公式,每个公式都是一个单独的段落,例如
\[\int_a^b{f\left(x\right)\mathrm{d}x}=\lim_{\left\|\Delta{x_i}\right\|\to 0}\sum_i{f\left(\xi_i\right)\Delta{x_i}}\]
除人工编号外,\XeLaTeX~无编号行间公式的标记见表~\ref{tab:eqtag_1},自动编号行间公式的标记间表~\ref{tab:eqtag_2}。
\begin{table}[htbp]
\caption{无编号行间公式的标记}\label{tab:eqtag_1}
\vspace{0.5em}\centering\wuhao
\begin{tabularx}{\textwidth}{cc}
\toprule
& 无编号\\
\midrule
单行公式 & \verb|\begin{displaymath}... \end{displaymath}| 或~\verb|\[...\]|\\
多行公式 & \verb|\begin{eqnarray*}... \end{eqnarray*}|\\
\bottomrule
\end{tabularx}
\end{table}

\begin{table}[htbp]
\caption{自动编号行间公式的标记}\label{tab:eqtag_2}
\vspace{0.5em}\centering\wuhao
\begin{tabularx}{\textwidth}{cc}
\toprule
& 自动编号\\
\midrule
单行公式 & \verb|\begin{equation}... \end{equation}|\\
多行公式 & \verb|\begin{eqnarray}... \end{eqnarray}|\\
\bottomrule
\end{tabularx}
\end{table}

另外,在自动编号的某行公式行尾添加标签~\verb|\nonumber|,可将该行转换为无编号形式。

行间多行公式需采用~\verb|eqnarray|~或~\verb|eqnarray*|~环境,它默认是一个列格式为~\verb|rcl|~的~3~列矩阵,并且中间列的字号要小一些,因此通常只将需要对齐的运算符号(通常为等号“=”)置于中间列。

\section{常见的数学式}
本节列举一些常见的数学式作为练习与未来使用的参考,每个函数都有其特别之处,请仔细观察研究。
读者可以依此为基础,在往后的写作过程中,逐渐累积更多有特殊型态的或符号的数学式,
只要这里出现过的,参照原使档一定写得出来。

\subsection{函数}
\begin{lstlisting}[language=TeX,numbers=none,frame=lrtb,keywords={begin},label=Binomial,caption=Binomial] 
$f(x)={n\choose x}p^x(1-p)^{1-x}, \;\; x=0,1,2,\cdots,n$ 
\end{lstlisting}
$f(x)={n\choose x}p^x(1-p)^{1-x}, \;\; x=0,1,2,\cdots,n$ 
   
\begin{lstlisting}[language=TeX,numbers=none,frame=lrtb,keywords={begin},label=Poisson,caption=Poisson] 
$f(x)=\frac{e^{-\lambda}\lambda^x}{x!}, \;\;  x=0,1,2,\cdots$ 
\end{lstlisting}
$f(x)=\frac{e^{-\lambda}\lambda^x}{x!}, \;\;  x=0,1,2,\cdots$
  
\begin{lstlisting}[language=TeX,numbers=none,frame=lrtb,keywords={begin},label=Gamma,caption=Gamma] 
$f(x)=\frac{1}{\Gamma(\alpha)\beta^\alpha}x^{\alpha-1} e^{-\frac{x}{\beta}}, \;\; x\geq 0$
\end{lstlisting}
$f(x)=\frac{1}{\Gamma(\alpha)\beta^\alpha}x^{\alpha-1}e^{-\frac{x}{\beta}}, \;\; x\geq 0$ 
  
\begin{lstlisting}[language=TeX,numbers=none,frame=lrtb,keywords={begin},label=Normal,caption=Normal] 
$f(x)=\frac{1}{\sigma\sqrt{2\pi}}e^{-\frac{(x-\mu)^2}{2\sigma^2}}, \;\;  -\infty < x < \infty $
\end{lstlisting}
$f(x)=\frac{1}{\sigma\sqrt{2\pi}}e^{-\frac{(x-\mu)^2}{2\sigma^2}}, \;\;  -\infty < x < \infty $
  
\begin{lstlisting}[language=TeX,numbers=none,frame=lrtb,keywords={begin},label=Int,caption=积分式与方程式编号] 
\begin{equation}\label{gamma}%..........label后的名称自订,代表该方程式
\int^\infty_0 x^{\alpha-1}e^{-\lambda x} dx = \frac{\Gamma(\alpha)}{\lambda^{\alpha}}
\end{equation}
\end{lstlisting}
\begin{equation}\label{gamma}%.................label后的名称自订,代表该方程式
\int^\infty_0 x^{\alpha-1}e^{-\lambda x} dx = \frac{\Gamma(\alpha)}{\lambda^{\alpha}}
\end{equation}
  
方程式 (\ref{gamma})是广义 $\Gamma$ 积分。\footnote{这里利用方程式标签(label)来引用方程式,编号将自动更新。}
 
\begin{lstlisting}[language=TeX,numbers=none,frame=lrtb,keywords={begin},label=Sqrt,caption=开根号] 
$$f(x)=\sqrt[3]{\frac {\displaystyle 4-x^{3}}{\displaystyle 1+x^{2}}}$$
\end{lstlisting}
$$f(x)=\sqrt[3]{\frac {\displaystyle 4-x^{3}}{\displaystyle 1+x^{2}}}$$
  
\begin{lstlisting}[language=TeX,numbers=none,frame=lrtb,keywords={begin},label=limit,caption=微分与极限(注意大刮号的使用)] 
$$f'(x)=\frac{df(x)}{dx}=\lim_{h\rightarrow 0} \left( \frac{f(x+h)-f(x)}{h} \right)$$
\end{lstlisting}  
$$f'(x)=\frac{df(x)}{dx}=\lim_{h\rightarrow 0}\left(\frac{f(x+h)-f(x)}{h}\right)$$

\begin{lstlisting}[language=TeX,numbers=none,frame=lrtb,keywords={begin},label=upanddown,caption=上下限的使用] 
$$\int_a^b f(x) dx \approx \lim_{n\rightarrow \infty}\sum_{k=1}^n f(x_k)\triangle x_k$$
\end{lstlisting} 
$$\int_a^b f(x) dx \approx \lim_{n\rightarrow \infty}\sum_{k=1}^n f(x_k)\triangle x_k$$
  
\begin{lstlisting}[language=TeX,numbers=none,frame=lrtb,keywords={begin},label=bast,caption=最佳化问题] 
$$\max_{\mathbf{u},\mathbf{u}^T\mathbf{u}=1} \mathbf{u}^T\Sigma_X\mathbf{u}$$
\end{lstlisting} 
$$\max_{\mathbf{u},\mathbf{u}^T\mathbf{u}=1} \mathbf{u}^T\Sigma_X\mathbf{u}$$
  
\begin{lstlisting}[language=TeX,numbers=none,frame=lrtb,keywords={begin},label=somesymbles,caption=几个符号]
$$\mathbf{e}=\mathbf{x}-\mathbf{x}_q=(I-P)\mathbf{x} \in V^{\perp}, \mbox{where}\; V\oplus V^{\perp}=R^p $$
\end{lstlisting} 
$$\mathbf{e}=\mathbf{x}-\mathbf{x}_q=(I-P)\mathbf{x} \in V^{\perp}, \mbox{where}\; V\oplus V^{\perp}=R^p $$


\subsection{矩阵与行列式}
矩阵或有规则排列的数学式或组合很常见,以下列举几种模式,请特别注意其使用的标签及一些需要注意的小地方。譬如,
\begin{enumerate}%[a)]
\item 矩阵的左右括号需各别加上。
\item 横行各项之间是以 $\&$ 区隔。
\item 除最后一行外,每行之末则加上换行指令$\backslash$ $\backslash$。
\item 使用array指令时,须加上选项以控制每一直栏内各数字或符号要居中排列、靠左或靠右。
\end{enumerate}
范例与注意事项:
\begin{enumerate}
  \item 左右方框刮号的使用及各直栏的对齐方式:
        $$ A = \left[
            \begin{array}{clr}
                a+b & mnop  & xy \\
                a+b & pn    & yz \\
                b+c & mp    & xyz
            \end{array} \right] $$
\begin{lstlisting}[language=TeX,numbers=none,frame=lrtb,keywords={begin}]
$$ A = \left[
\begin{array}{clr}
a+b & mnop  & xy \\
a+b & pn    & yz \\
b+c & mp    & xyz
\end{array} \right] $$
\end{lstlisting} 

\item 左右圆框刮号的使用及各式点状:
$$ A=\left(
\begin{array}{cccc}
a_{11} & a_{12} & \cdots & a_{1n}\\
a_{21} & a_{22} & \cdots & a_{2n}\\
\vdots & \vdots & \ddots & \vdots\\
a_{n1} & a_{n2} & \cdots & a_{nn}
\end{array} \right) $$
\begin{lstlisting}[language=TeX,numbers=none,frame=lrtb,keywords={begin}]
$$ A=\left(
\begin{array}{cccc}
a_{11} & a_{12} & \cdots & a_{1n}\\
a_{21} & a_{22} & \cdots & a_{2n}\\
\vdots & \vdots & \ddots & \vdots\\
a_{n1} & a_{n2} & \cdots & a_{nn}
\end{array} \right) $$
\end{lstlisting} 

\item 排列整齐的符号:
$$ \begin{array}{clr}\\
a+b+c   & m+n & xy \\
a+b     & p+n & yz \\
b+c     & m-n & xz
\end{array} $$
\begin{lstlisting}[language=TeX,numbers=none,frame=lrtb,keywords={begin}]
$$ \begin{array}{clr}\\
a+b+c   & m+n & xy \\
a+b     & p+n & yz \\
b+c     & m-n & xz
\end{array} $$
\end{lstlisting}

\item 等号对齐的函数组合(不编号)
\begin{eqnarray*}
b_1 &=& d_1+c_1 \\
a_2 &=& c_2+e_2
\end{eqnarray*}
\begin{lstlisting}[language=TeX,numbers=none,frame=lrtb,keywords={begin}]
\begin{eqnarray*}
b_1 &=& d_1+c_1 \\
a_2 &=& c_2+e_2
\end{eqnarray*}
\end{lstlisting}

\item 等号对齐的函数组合(编号在最后一行)
\begin{eqnarray}
\nonumber b_1 &=& d_1+c_1 \\
a_2 &=& c_2+e_2
\end{eqnarray}
\begin{lstlisting}[language=TeX,numbers=none,frame=lrtb,keywords={begin}]
\begin{eqnarray}
\nonumber b_1 &=& d_1+c_1 \\
a_2 &=& c_2+e_2
\end{eqnarray}
\end{lstlisting}

\item 使用巨集 amsmath 的指令 align(控制编号在第一行)
\begin{align}
b_1 &= d_1+c_1\\
a_2 &= c_2+e_2 \notag
\end{align}
\begin{lstlisting}[language=TeX,numbers=none,frame=lrtb,keywords={begin}]
\begin{align}
b_1 &= d_1+c_1\\
a_2 &= c_2+e_2 \notag
\end{align}
\end{lstlisting}

\item 两组数学式分别对齐
\begin{align}
\alpha_1 &= \beta_1+\gamma_1+\delta_1, &a_1 &= b_1+c_1\\
\alpha_2 &= \beta_2+\gamma_2+\delta_2, &a_2 &= b_2+c_2
\end{align}
\begin{lstlisting}[language=TeX,numbers=none,frame=lrtb,keywords={begin}]
\begin{align}
\alpha_1 &= \beta_1+\gamma_1+\delta_1, &a_1 &= b_1+c_1\\
\alpha_2 &= \beta_2+\gamma_2+\delta_2, &a_2 &= b_2+c_2
\end{align}
\end{lstlisting}

\item 编号在中间(split指令环境)
\begin{equation}
\begin{split}
\alpha_1 &= \beta_1+\gamma_1\\
\alpha_2 &= \beta_2+\gamma_2
\end{split}
\end{equation}
\begin{lstlisting}[language=TeX,numbers=none,frame=lrtb,keywords={begin}]
\begin{equation}
\begin{split}
\alpha_1 &= \beta_1+\gamma_1\\
\alpha_2 &= \beta_2+\gamma_2
\end{split}
\end{equation}
\end{lstlisting}

\item 只是居中对齐的数学式组(gather指令环境)
\begin{gather}
\alpha_1 + \beta_1\notag\\
\alpha_2 + \beta_2 + \gamma_2\notag
\end{gather}
\begin{lstlisting}[language=TeX,numbers=none,frame=lrtb,keywords={begin}]
\begin{gather}
\alpha_1 + \beta_1\notag\\
\alpha_2 + \beta_2 + \gamma_2\notag
\end{gather}
\end{lstlisting}

\item 长数学式的表达(注意第二行加号的位置)
\begin{align}
y   &= x_1 + x_2 + x_3 \notag\\
    &\quad + x_4 + x_5
\end{align}
\begin{lstlisting}[language=TeX,numbers=none,frame=lrtb,keywords={begin}]
\begin{align}
y   &= x_1 + x_2 + x_3 \notag\\
	&\quad + x_4 + x_5
\end{align}
\end{lstlisting}
\end{enumerate}


\subsection{其他}

$$X_{n} \stackrel{d}{\longrightarrow} X$$\\
\begin{lstlisting}[language=TeX,numbers=none,frame=lrtb,keywords={begin}]
$$X_{n} \stackrel{d}{\longrightarrow} X$$
\end{lstlisting}

$$\overbrace{X_{1} + \ldots + \underbrace{X_{15} + \ldots + X_{30}}}$$\\
\begin{lstlisting}[language=TeX,numbers=none,frame=lrtb,keywords={begin}]
$$\overbrace{X_{1} + \ldots + \underbrace{X_{15} + \ldots + X_{30}}}$$
\end{lstlisting}

\begin{equation*}
G = \left\{\begin{array}{l}
CLASS\#1 \;\;\mbox{if} \;\; \hat{\beta}^T\bf{x} \leq 0 \\
CLASS\#2 \;\;\mbox{if} \;\; \hat{\beta}^T\bf{x} > 0
\end{array}\right.
\end{equation*}\\
\begin{lstlisting}[language=TeX,numbers=none,frame=lrtb,keywords={begin}]
\begin{equation*}
G = \left\{\begin{array}{l}
CLASS\#1 \;\;\mbox{if} \;\; \hat{\beta}^T\bf{x} \leq 0 \\
CLASS\#2 \;\;\mbox{if} \;\; \hat{\beta}^T\bf{x} > 0
\end{array}\right.
\end{equation*}
\end{lstlisting}

以equation或align排版时,数学式会自动编上号码。文稿其他地方若要引述某数学式,
可先以$\backslash$label指令加上标签,再使用$\backslash$ref指令引述。
如此一来若排版文稿须反覆修改,使用$\backslash$label 与$\backslash$ref 指令可以「自动对焦」不会出错。


%\section{可自动调整大小的定界符}
%若在左右两个定界符之前分别添加命令~\verb|\left|~和~\verb|\right|,则定界符可根据所包围公式大小自动调整其尺寸,这可从式(\ref{nodelimiter})和式(\ref{delimiter})中看出。
%\begin{equation}\label{nodelimiter}
%(\sum_{k=\frac12}^{N^2})
%\end{equation}
%\begin{equation}\label{delimiter}
%\left(\sum_{k=\frac12}^{N^2}\right)
%\end{equation}
%式(\ref{nodelimiter})和式(\ref{delimiter})是在~\XeLaTeX~中分别输入如下代码得到的。
%\begin{verbatim}
%(\sum_{k=\frac12}^{N^2})
%\left(\sum_{k=\frac12}^{N^2}\right)
%\end{verbatim}
%\verb|\left|~和~\verb|\right|~总是成对出现的,若只需在公式一侧有可自动调整大小的定界符,则只要用“.”代替另一侧那个无需打印出来的定界符即可。
%
%若想获得关于此部分内容的更多信息,可参见~\href{http://tug.ctan.org/cgi-bin/ctanPackageInformation.py?id=voss-mathmode}{Math mode}~文档的第~8~章“Brackets, braces and parentheses”。

\section{数学重音符号}
数学重音符号通常用来区分同一字母表示的不同变量,输入方法如下(需要调用~\verb|amsmath|~宏包):

\vspace{0.5em}{\noindent\wuhao\begin{tabularx}{\textwidth}{Xc|Xc|Xc}
 \verb|\acute| & $\acute{a}$ & \verb|\mathring| & $\mathring{a}$ & \verb|\underbrace| & $\underbrace{a}$ \\
 \verb|\bar| & $\bar{a}$ & \verb|\overbrace| & $\overbrace{a}$ & \verb|\underleftarrow| & $\underleftarrow{a}$ \\
 \verb|\breve| & $\breve{a}$ & \verb|\overleftarrow| & $\overleftarrow{a}$ & \verb|\underleftrightarrow| & $\underleftrightarrow{a}$ \\
 \verb|\check| & $\check{a}$ & \verb|\overleftrightarrow| & $\overleftrightarrow{a}$ & \verb|\underline| & $\underline{a}$ \\
 \verb|\dddot| & $\dddot{a}$ & \verb|\overline| & $\overline{a}$ & \verb|\underrightarrow| & $\underrightarrow{a}$ \\
 \verb|\ddot| & $\ddot{a}$ & \verb|\overrightarrow| & $\overrightarrow{a}$ & \verb|\vec| & $\vec{a}$ \\
 \verb|\dot| & $\dot{a}$ & \verb|\tilde| & $\tilde{a}$ & \verb|\widehat| & $\widehat{a}$ \\
 \verb|\grave| & $\grave{a}$ & \verb|\underbar| & $\underbar{a}$ & \verb|\widetilde| & $\widetilde{a}$ \\
 \verb|\hat| & $\hat{a}$
\end{tabularx}}\vspace{0.5em}
当需要在字母~$i$~和~$j$~的上方添加重音符号时,为了去掉这两个字母顶上的小点,这两个字母应该分别改用~\verb|\imath|~和~\verb|\jmath|。

如果遇到某些符号不知道该采用什么命令能输出它时,则可通过~\href{http://detexify.kirelabs.org/classify.html}{Detexify$^2$~网站}来获取符号命令。若用鼠标左键在此网页的方框区域内画出你所要找的符号形状,则会在网页右方列出和你所画符号形状相近的~5~个符号及其相对应的~\XeLaTeX~输入命令。若所列出的符号中不包括你所要找的符号,还可通过点击“Select from the complete list!”的链接以得分从低到高的顺序列出所有符号及其相对应的~\XeLaTeX~输入命令。

最后,建议大家还以~\href{http://tug.ctan.org/cgi-bin/ctanPackageInformation.py?id=voss-mathmode}{Math mode}~这篇~pdf~文档作为主要参考。若要获得最为标准、美观的数学公式排版形式,可以查查文档中是否有和你所要的排版形式相同或相近的代码段,通过修改代码段以获得你所要的数学公式排版形式。

