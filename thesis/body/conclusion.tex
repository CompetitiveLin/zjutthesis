% !Mode:: "TeX:UTF-8"

\chapter{总结}

\section{完成的工作}

本文论述魔方求解算法的发展历程以及国内外魔方复原系统的研究现状,接着分别从硬件和软件层面详细阐述如何设计并实现了该魔方复原系统,并且在一定程度上具有教育意义和商业价值。本系统完成的工作如下所示:

(1)设计建模并搭建了魔方复原系统的框架,设计PCB电路板以达到系统高度集成化的目的。

(2)对上下位机进行编程,在系统完成的基础上优化了魔方复原的方式,并行还原的方式实现了魔方复原步骤的最少化。

(3)设计并开发了系统的可视化界面,增加人机交互功能,例如颜色矫正,还原可行性验证等。

\section{存在的问题及下一步工作}

本魔方复原系统结合已知最优的还原算法,并自主设计实现并行操作的优化算法,在复原操作方面可能已经达到国内外最新发展水平。但是,本系统在诸多方面仍然存在着明显不足,例如识别魔方块颜色的速度与准确率等,仍然会存在一些无法避免的错误,也正是因为这些错误,往往对系统的运行结果产生重大的影响。因此,在后续更进一步的研究中,本系统将会在以下几个方面着重进行其完善的工作:

(1)完善图像识别算法,使系统免于不同环境因素尤其是不同光照条件的影响。目前本算法的鲁棒性不够强,并不能完全准确识别所有不同环境情况下的魔方块颜色。与此同时,在测试系统功能期间也可以增加多种复杂环境下的情况,使其充分考虑环境因素。初步拟定设计环境适应性更强的图像识别算法。

(2)降低识别过程中的操作繁琐程度。就目前的系统而言,使用三个不同方位的摄像头需要通过两次的底面180°的旋转才能捕捉到所有魔方面的所有魔方块颜色,在识别的过程中仍存在着较高的操作繁琐程度。

(3)设计更美观的可视化界面。目前,本系统的可视化界面仅仅是由数个简单的按钮和标签栏组成,并不是特别美观,也无操作提示,系统界面比较一般。在后续的工作中将融入其他风格的系统界面元素,使之更加美观。 

